%------------------------------------------
\IEEEraisesectionheading{\section{Introduction}\label{sec:intro}}
%------------------------------------------
\IEEEPARstart{C}{lustering} is a fundamental method for understanding and interpreting data that
seeks to partition input objects into groups, known as \emph{clusters}, such
that objects within a cluster are similar to each other, and objects in
different clusters are not.  A clustering formulation called $k$-means is
simple, intuitive, and widely used in practice.  Given a set of points $S$ in a
Euclidean space and a parameter $k$, the objective of $k$-means is to partition
$S$ into $k$ clusters in a way that minimizes the sum of the squared distance
from each point to its cluster center.

In many cases, the input data is not available all at once but arrives as a
continuous, possibly unending, sequence.  This variant, known as \emph{streaming
$k$-means clustering}, requires an algorithm to maintain enough state to be
able to incrementally update the clusters as more data arrive.  Furthermore,
when a query is posed, the algorithm is required to return $k$ cluster centers,
one for each cluster, for the data observed so far.

Prior work on streaming $k$-means (e.g.~\cite{AMR+12,AJM09,GMM+03,SWM11}) has
mainly focused on optimizing the memory requirements, leading to algorithms with
provable approximation quality that only use space polylogarithmic in the input
stream's size~\cite{AMR+12,AJM09}.  However, these algorithms require an
expensive computation to answer a query for cluster centers.  This can be a
serious problem for applications that need answers in (near) real-time, such as
in network monitoring and sensor data analysis.  Our work aims at improving the
clustering query-time while keeping other desirable properties of current
algorithms.

To understand why current solutions to streaming $k$-means clustering have a high query runtime, let us review the
framework they use. At a high level, an incoming data stream $\stream$ is divided into smaller ``chunks''
$\stream_1,\stream_2,\ldots,$. Each chunk is summarized into a compact
representation, known as a ``coreset'' (for example, see~\cite{HM04}). The
resulting coresets may still not all fit into the memory of the processor.
Hence, multiple coresets are further merged recursively into higher-level
coresets, forming a hierarchy of coresets, or a ``coreset tree''.  When a query
arrives, all active coresets in the coreset tree are merged together, and a
clustering algorithm such as $\kmpp$~\cite{AV07} is applied on the result,
outputting $k$ cluster centers. The query time is consequently proportional to
the number of coresets that need to be merged together. In prior algorithms, the
total size of all these coresets could be as large as the whole memory itself,
which causes the query time to often be prohibitively high.


\subsection{Our Contributions}
We present three algorithms (\cc{}, \rcc{}, and \hybrid{}) for
streaming $k$-means clustering that asymptotically and practically improve upon
prior algorithms' response time of a query while retaining guarantees on memory
efficiency and solution quality of the current state-of-the-art.  We provide
theoretical analysis, as well as extensive empirical evaluation, of the proposed
algorithms.

At the heart of our algorithms is the idea of ``coreset caching'' that to our
knowledge, has not been used before in streaming clustering. It works by reusing
coresets that have been computed during previous (recent) queries to speedup the
computation of a coreset for the current query. In this way, when a query
arrives, the algorithm has no need to combine all coresets currently in memory;
it only needs to merge a coreset from a recent query (stored in the coreset
cache) along with coresets of points that arrived after this
query.

%------------------------------------------
%--------------------------------------------------------------------------------------------------
\begin{table*}[ht]
{
\small
\centering
\begin{tabular}{|c| c| c| c| c|}
\hline
Name & Query Cost & Update Cost & Memory Used & Accuracy: Coreset level  \\
           & (per point)   & (per point)    & & returned at query  \\
\hline
Coreset Tree ($\ct$) & $O\left( \frac{kdm}{q} \cdot \frac{r \log N}{\log r}\right)$ &  $O(dm)$ & $O\left(dm \cdot \frac{r \log N}{\log r}\right)$ & $\log N / \log r$ \\
\hline
Cached Coreset Tree ($\cc$)&  $O\left(\frac{kdm}{q} \cdot r \right)$ & $O(dm)$ & $O\left(dm \cdot \frac{r\log N}{\log r} \right)$ & $2  \log N / \log r$\\
\hline
Recursive Cached Coreset Tree ($\rcc$) & $O\left(\frac{kdm}{q} \cdot \log \log N \right)$ & $O(dm \log \log N)$ & $O(dm N^{1/8})$ & $O(1)$ \\
\hline
Online Coreset Cache ($\hybrid$)  & usually $O(1)$ worst case $O\left(\frac{kdm}{q} \cdot r \right)$ & $O(dm)$ & $O\left(dm \cdot \frac{r\log N}{\log r} \right)$  & $2 \log N / \log r$  \\
\hline
\end{tabular}
\smallskip
}
\caption{The time and accuracy of different clustering algorithms. For algorithms other than $\ct$, we assume answering queries by using the coreset cache.}
% $k$ is the
%   number of centers desired, $d$ is the dimension of a data point, $m$ is the
%   size of a coreset (in practice, this is a constant factor times $k$), $r$ is a
%   parameter used for $\cc$ and $\ct$, showing the ``merge degree" of the coreset
%   tree, $q$ is the number of points between two queries. The ``level" of a
%   coreset is indicative of the number of recursive merges of prior coresets to
%   arrive at this coreset. Smaller the level, more accurate is the coreset. For
%   example, a batch algorithm that sees the entire input can return a coreset at
%   level $0$.}
\label{table:summary}
\end{table*}
%--------------------------------------------------------------------------------------------------

%------------------------------------------

Our theoretical results are summarized in Table~\ref{table:summary}.
Throughout, let $n$ denote the number of points observed in the stream so far.
We measure an algorithm's performance in terms of both running time (further
separated into query and update) and memory consumption.  The query cost, stated
in terms of $q$, represents the expected amortized cost per input point assuming
that the total number of queries does not exceed $n/q$---or that the average
interval between queries is $\Omega(q)$. The update cost is the average (i.e.,
amortized) per-point processing cost, taken over the entire stream.  The memory
cost is expressed in terms of words assuming that each point is $d$-dimensional
and can be stored in $O(d)$ words.  Furthermore, let $m$ denote a user-defined
parameter that determines the coreset size ($m$ is set independent of $n$ and is
often $O(k)$ in practice); $r$ denote a user-defined parameter that sets the
merge degree of the coreset tree; and $N = n/m$ be the number of ``base
buckets.''

In terms of accuracy, each of our algorithms provides a provable
$O(\log k)$-approximation to the optimal $k$-means solution---that is, the
quality is comparable, up to constants, to what we would obtain if we simply
stored all the points so far in memory and ran an (expensive) batch $\kmpp$
algorithm at query time.  Further scrutiny reveals that for the same target
accuracy (holding constants in the big-$O$ fixed), our simplest algorithm
``Cached Coreset Tree'' ($\cc$) improves the query runtime of a current
state-of-the-art, \ct{}\footnote{$\ct$ is essentially the \skmpp
  algorithm~\cite{AMR+12} and~\cite{AJM09} except it has a more flexible rule
  for merging coresets.}, by a factor of $O(\log N)$---at the expense of a
(small) constant factor increase in memory usage.  If more flexibility in
tradeoffs among the parameters is desired, coreset caching can be applied
recursively.  Using this scheme, the ``Recursive Cached Coreset Tree'' ($\rcc$)
algorithm can be configured so that it has a query runtime that is a factor of
$O(r)$ times faster than \ct, and yields better solution quality
than \ct{}, at the cost of a small polynomial factor increase in memory.

In practice, a simple sequential streaming clustering algorithm, due to
MacQueen~\cite{MacQueen67}, is known to be fast but lacks good theoretical
properties.  We derive an algorithm, called \hybrid{}, that combines ideas from
\cc and sequential streaming to further enhance clustering query runtime while
keeping the provable clustering quality of \rcc and \cc.

We also present an extensive empirical study of the proposed algorithms, in
comparison to the state-of-the-art $k$-means algorithms, both batch and
streaming.  The results show that our algorithms yield substantial speedups
(5--100x) in query runtime and total time, and match the accuracy of \skmpp for
a broad range of datasets and query arrival frequencies.


%------------------------------------------
\subsection{Related Work}
\label{sec:related}
% ------------------------------------------

When all input is available at the start of computation (batch setting), Lloyd's
algorithm~\cite{Lloyd82}, also known as the \km algorithm, is a simple,
widely-used algorithm.  Although it has no quality guarantees, heuristics such
as \kmpp~\cite{AV07} can generate a starting configuration such that Llyod's
algorithm will produce provably-good clusters.

In the streaming setting, \cite{MacQueen67} is the earliest streaming
$k$-means method, which maintains the current cluster centers and applies one
iteration of Lloyd's algorithm for every new point received.  Because it is fast
and simple, this sequential algorithm is commonly used in practice (e.g., Apache Spark
mllib~\cite{MBY+15}).  However, it cannot provide any guarantees on the
quality~\cite{KMN+04}.  BIRCH~\cite{ZRL96} is a streaming clustering method
based on a data structure called the CF Tree; it produces cluster centers
through agglomerative hierarchical clustering on the leaf nodes of the
tree. CluStream\cite{AHW+03} constructs ``microclusters'' that summarize subsets
of the stream, and further applies a weighted \km algorithm on the
microclusters. STREAMLS~\cite{GMM+03} is a divide-and-conquer method based on
repeated application of a bicriteria approximation algorithm for clustering. A
similar divide-and-conquer algorithm based on \kmpp is presented
in~\cite{AJM09}. Invariably, these methods have high query-processing cost and
are not suitable for applications that require fast query response.  In
particular, at the time of query, they require merging multiple data structures,
followed by an extraction of cluster centers, which is expensive.

Har-Peled and Mazumdar~\cite{HM04} present coresets of size
$O(k \eps^{-d} \log n)$ for summarizing $n$ points \km, and also show how to
use the merge-and-reduce technique based on the Bentley-Saxe
decomposition~\cite{BS80} to derive a small-space streaming algorithm using
coresets. Further work~\cite{HK07, FL11, FSS13} reduced the size of a \km coreset
to $O(k/\eps^2)$. A close cousin to ours, \skmpp~\cite{AMR+12}
(essentially the \ct{} scheme) is a streaming $k$-means clustering algorithm
that uses the merge-and-reduce technique along with \kmpp to generate a
coreset. Our work improves on \skmpp w.r.t. query runtime.

\noindent\textbf{Roadmap:} We present preliminaries in Section~\ref{sec:prelim},
background for streaming clustering in Section~\ref{sec:background} and then the
algorithms $\cc$, $\rcc$, and $\hybrid$ in Section~\ref{sec:algorithm}, along
with their proofs of correctness and quality guarantees. We then present
experimental results in Section~\ref{sec:expts}.
